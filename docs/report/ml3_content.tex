
The main purpose of this project is to improve data mining techniques by applying
methods related to computational linguistics and to try out work on real data.
I used the data of 
\href{http://2012.msrconf.org/challenge.php}{the~MSR 2012 challenge.} The challenges goal is to simple find out interesting information about Android previous development.

This project extends project which I have done together with Peteris Nikiforovs. 
We have explored several classifiers used for data mining.
I decided to pick five most effective classifiers from previous project and apply 
as many techniques as possible from a computational linguistic class. 
In fact, as we finished the data mining project with Peter, we immediately saw that
we based our classification on the our data with textual attributes.
Applying text mining techniques should improve understanding of the data and hopefully
improve the results of the classification experiments.

I selected several successful experiments and I have applied different transformation techniques to improve our results
and to explore the techniques, which transformed raw text entries to format acceptable for classification.

As experiments, I intentionally picked up the one with Support Vector Model Classifier and Decision Tree classifier,
because we used them in the labs and we have learned about them in the lectures. As the last classifier, I picked up
k-NN (k Nearest Neighbour) classifier, because k-NN is efficient and also easy to understand.
Learning phase is basically omitted and the classification of unseen item is perform as majority voting
of {\it k} nearest labels from training data.

I have slightly improved precision of previous experiments, but mainly
I have learnt how the classifiers relates to different preprocessing techniques. 

Note that we run the experiments in Rapid Miner, an open source software, which encapsulates 
lot of open source algorithms for data mining including Weka. 
