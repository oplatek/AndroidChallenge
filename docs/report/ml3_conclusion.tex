 
As can be seen, it is possible to automatically classify bugs by their content more than half of the time. All of the trialled algorithms could reach the 40\% accuracy barrier, however, the performance between these algorithms varies. 

The goal of this project was to complete the overview of precision among different classifiers by comparing their precisions depending on textual preprocessing.

The first conclusion from this project is that we (Peter and I) had originally chosen quite good settings. 
The examined techniques except new tokenizer did not improve the results. The tokenizer improved the results approximately by 2\%.

Another more important conclusion is that different vectorising 
techniques are suitable for different classifiers.
The binary occurrence is at one side computed fast, but does not work well with {\it k-NN} and {\it SVM}. Especially, for {\it k-NN} the binary occurrence is really bad, because for a quite big {\it k} the distance between can be just 0 or 1 for sparse vectors.   

Last but not least, I used profesional tool which is reliable
and I am going to use it future too.
% subsection Classification of bugs bugs by component (end)
